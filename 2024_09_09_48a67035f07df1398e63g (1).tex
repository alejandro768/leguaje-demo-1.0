% This LaTeX document needs to be compiled with XeLaTeX.
\documentclass[10pt]{article}
\usepackage[utf8]{inputenc}
\usepackage{amsmath}
\usepackage{amsfonts}
\usepackage{amssymb}
\usepackage[version=4]{mhchem}
\usepackage{stmaryrd}
\usepackage{graphicx}
\usepackage[export]{adjustbox}
\graphicspath{ {./images/} }
\usepackage[fallback]{xeCJK}
\setCJKmainfont{Noto Serif CJK JP}

\setmainlanguage{spanish}
\setmainfont{CMU Serif}

\title{Implementacion del analizador lexico }

\author{}
\date{}


\begin{document}
\maketitle
\begin{enumerate}
  \item Analisis del código\\
1.1) Definición de los tokens
\end{enumerate}

\begin{verbatim}
tokens = (
    'CUANDO', 'MIENTRAS', 'SI', 'SI_NO', 'IMPRIMIR', 'ASIGNAR', 'O_SI', 'TAMBIEN',
    'ENTERO', 'FLOTANTE', 'DOBLE', 'BOOLEANO', 'CADENA_CARACTER', 'CARACTER',
    'SUMA', 'MENOS', 'MULTIPLICACION', 'DIVISION',
    'MAYOR', 'MENOR', 'MENOR_O_IGUAL', 'MAYOR_O_IGUAL',
    'IGUAL_COMPARACION', 'DESIGUALDAD',
    'COMENTARIO_DE_BLOQUE', 'PARENTESIS_INICIO', 'PARENTESIS_FIN', 'SALTO_LINEA'
)
\end{verbatim}

1.2) Las expresiones regulares para los tokens simples como operadores aritméticos ( + , -, \textit{, /) y comparaciones ( $>,<,<=,>=,==$ ! !=)\\
t\_SUMA $=r^{\prime} \backslash+{ }^{\prime}$\\
t\_MENOS $=r^{\prime}-{ }^{\prime}$\\
t\_MULTIPLICACION $=r^{\prime} \backslash^{*}$\\
t\_DIVISION = r' /'\\
t\_MAYOR $=r^{\prime}>^{\prime}$\\
t\_MENOR $=r^{\prime}<^{\prime}$\\
t\_MENOR\_0\_IGUAL $=r^{\prime}<='$\\
t\_MAYOR\_0\_IGUAL $=r^{\prime}>='$\\
t\_IGUAL\_COMPARACION $=r^{\prime}=='$\\
t\_DESIGUALDAD $=r^{\prime}!='$\\
$\\
1.3) Se define la regla para reconocer y omitir comentarios de bloque delimitados por /} y */.\\
def t\_COMENTARIO\_DE\_BLOQUE $(\mathrm{t}):$\\
$r^{\prime} / \backslash^{*} . * ?$ ? $^{\prime} /{ }^{\prime}$\\
pass \# Los comentarios de bloque se ignoran\\
1.4) Se definen reglas para palabras clave específicas del lenguaje.

\includegraphics[max width=\textwidth, center]{2024_09_09_48a67035f07df1398e63g-2}\\
1.5) Se definen reglas para los tipos de datos (ENTERO, FLOTANTE, DOBLE, BOOLEANO, CADENA\_CARACTER, CARACTER).

\includegraphics[max width=\textwidth, center]{2024_09_09_48a67035f07df1398e63g-3}\\
1.6) La regla para SALTO\_LINEA es usada para incrementar el número de línea en el lexer cuando se encuentra la palabra finL.\\
def t\_SALTO\_LINEA $(\mathrm{t}):$\\
$r^{\prime}$ finL'\\
t.lexer.lineno t= 1 \# Incrementa el número de línea en 1 return $\mathbf{t}$\\
1.7) Construcción del Lexer: Se construye el lexer con lex.lex().\\
1.7.1)Prueba del Lexer: Se define un conjunto de datos de prueba, se alimenta al lexer y se imprimen los tokens reconocidos.

\includegraphics[max width=\textwidth, center]{2024_09_09_48a67035f07df1398e63g-4}\\
2) Tabal de tokens y expresiones regulares

\begin{center}
\begin{tabular}{|c|c|c|}
\hline
Token & Expresión Regular & Descripción \\
\hline
CUANDO & cuando & \begin{tabular}{l}
Palabra clave para \\
iniciar un bloque similar \\
a while \\
\end{tabular} \\
\hline
MIENTRAS & mientras & \begin{tabular}{l}
Palabra clave para un \\
bucle similar a for \\
\end{tabular} \\
\hline
SI & si & \begin{tabular}{l}
Palabra clave para una \\
condición (if) \\
\end{tabular} \\
\hline
SI\_NO & si\_no & \begin{tabular}{l}
Palabra clave para una \\
alternativa (else) \\
\end{tabular} \\
\hline
IMPRIMIR & imprimir & \begin{tabular}{l}
Palabra clave para \\
imprimir valores \\
\end{tabular} \\
\hline
ASIGNAR & asignar & \begin{tabular}{l}
Palabra clave para \\
asignar valores a \\
variables \\
\end{tabular} \\
\hline
O\_SI & o\_si & \begin{tabular}{l}
Palabra clave para una \\
alternativa adicional \\
(elif) \\
\end{tabular} \\
\hline
TAMBIEN & tambien & \begin{tabular}{l}
Palabra clave para \\
operaciones adicionales \\
\end{tabular} \\
\hline
RETORNAR & retorno & \begin{tabular}{l}
Palabra clave para \\
retornar valores \\
\end{tabular} \\
\hline
FINL & finL & \begin{tabular}{l}
Palabra clave para el fin \\
de una línea o bloque \\
\end{tabular} \\
\hline
ENTERO & $[0-9]^{*}$ & Tipo de dato entero \\
\hline
FLOTANTE & $[0-9]^{*}+()+.[0-9]^{*}$ & Tipo de dato flotante \\
\hline
DOBLE & $[0-9] *+()+.[0-9] *$ & \begin{tabular}{l}
Tipo de dato de doble \\
precisión \\
\end{tabular} \\
\hline
BOOLEANO & [verdad | falso] & \textbackslash bverdad \textbackslash bfalso` \\
\hline
CADENA\_CARACTER & $(")+[\bullet]^{*}+\left({ }^{\prime}\right)$ & \begin{tabular}{l}
Tipo de dato cadena de \\
caracteres \\
\end{tabular} \\
\hline
CARACTER & (")+[・]+(") & Tipo de dato carácter \\
\hline
SUMA & + & Operador de suma \\
\hline
MENOS & - & Operador de resta \\
\hline
MULTIPLICACION & * & \begin{tabular}{l}
Operador de \\
multiplicación \\
\end{tabular} \\
\hline
DIVISION & 1 & Operador de división \\
\hline
MAYOR & $>$ & Operador de mayor que \\
\hline
MENOR & $<$ & Operador de menor que \\
\hline
MENOR\_O\_IGUAL & $<=$ & \begin{tabular}{l}
Operador de menor o \\
igual que \\
\end{tabular} \\
\hline
MAYOR\_O\_IGUAL & $=>$ & \begin{tabular}{l}
Operador de mayor o \\
igual que \\
\end{tabular} \\
\hline
IGUAL\_COMPARACION & $==$ & Operador de igualdad \\
\hline
\end{tabular}
\end{center}\\
\begin{tabular}{|c|c|c|}
\hline
DESIGUALDAD & $!=$ & \begin{tabular}{l}
Operador de \\
desigualdad \\
\end{tabular} \\
\hline
COMENTARIO\_DE\_BLOQUE & ( /* ) + [ ・ ]* + ( */ ) & \begin{tabular}{l}
Comentario de bloque \\
delimitado por /* y */ \\
\end{tabular} \\
\hline
PARENTESIS\_INICIO & ( & Paréntesis de apertura \\
\hline
PARENTESIS\_FIN & ) & Paréntesis de cierre \\
\hline
SALTO\_LINEA & finL & \begin{tabular}{l}
Marca el final de una \\
línea o bloque \\
\end{tabular} \\
\hline
\end{tabular}


\end{document}